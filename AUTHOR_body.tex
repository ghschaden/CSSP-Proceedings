
% AUTHOR_body.tex
% Prepared by C. Pinon (cjpinon@implicature.xyz) for EISS 13 (see
% https://implicature.xyz/eiss13/)
% 2020-02-02

% Please use this file (which you should rename) for the main ("body")
% part of your paper. Be sure to compile the master file, not this one.

% Put your main text here. Recall that the references are in a separate
% file. And recall also that "\begin{document}" and "\end{document}"
% have already been inserted, which means that you can begin immediately
% with the first section.

\section{Introduction}\label{sec:1}

In this paper, we will show how semantics and pragmatics reduce to
syntax (or the other way around, see \citealt{Kamp:1973,Searle:1964}) \ldots 

Yes you can śačĉḥ -- ţŷpe şōmê wild Üñïçôđė.

For examples, use \textbf{linguex}, and for glossing morphological categories, please refer to the Leipzig Glossing Rules.

\exg. 
win er der Namen Gottes het usgsprochn-a ghabe\\
when he the name.\textsc{m.sg} God.\textsc{gen} has pronounced-\textsc{m.sg} had\\
%\glt 
`once he had pronounced the name of God'

For examples with several subexamples:

\ex. \label{ex:1}\a. \label{ex:2}First example
\b. Second
\bg. Dies ist ein schönes Beispiel mit einer Glose.\\
This is a beautiful example with a gloss.\\
\a. `Some translation'
\b. `Some other'
\z.
\b. Fourth

You can refer back to examples like \ref{ex:1} and \ref{ex:2}.

In order to make trees, use the \textbf{forest} package.

% LaTeX doesn't care how you arrange the tree in the source code, but for your won sake, you want probably want to write something as follows:
\ex. \begin{forest}
[S 
  [DP [D [ Ce\\\footnotesize{This} ] ]
      [NP [AP [A [ petit\\\footnotesize{small} ] ] ]
          [N [ arbre\\\footnotesize{tree} ] ] ] ]
  [VP [V [ est\\\footnotesize{is} ] ]
      [AP [AdvP [Adv [ très\\\footnotesize{very} ] ] ]
          [A [ joli\\\footnotesize{nice} ] ] ] ] ]
\end{forest}

You can also make a more involved example, with arrows:

\ex.  \label{ex:rp-structure} 
\footnotesize{%
\begin{forest} for tree={l=10mm, l sep=0}
   [AuxP, name=top 
      [beP [RP 
              [DP [John] ] 
              [R$'$ [withP  
                       [RP [DP [hair] ] 
                           [R$'$ [AP [VP ] 
                                     [A [ colored, name=cellar ] ] ] 
                                 [R] ] ] 
                        [with, name=source] ] 
                    [R, name=c1] ] ] 
            [be, name=c2]   ]  
      [Aux [ has ] ] ]
    % and now, we draw some arrows  
   \draw[->,color=red,very thick,dotted] (source) to[out=east,in=south east] (c1);
   \draw[->,color=blue,ultra thick,dashed] (c1) to[out=east,in=south east] (c2);
   \draw[-{Latex[length=2.5mm]},color=purple,ultra thick] (cellar) to[out=east,in=east]  
                        node [below, sloped] (TextNode2) {because why the hell not} (top);
\end{forest}}

\citet[451]{Szadrowsky:1936} claims something, but I am not sure whether \citet[70]{Kamp:1973} would agree (or \citealt[50]{Searle:1964}, by that matter).

For avms, we use \textbf{langsci-avm}.

\ex. \avm{[ attr1 & \1\\
attr2 & \2[attr3 & val3\\
attr4 & val4] ]}

You can combine \textbf{forest} and \textbf{langsci-avm}:

\ex. \begin{forest}
[A [B] [{\avm{[attr1 & val1\\
attr2 & val2\\
attr3 & val3]}} ] ]
\end{forest}l

And finally, for tables, we use the \textbf{booktabs}-environment, as illustrated in table \ref{tab:article-systems}. 



\textbf{Question: do we insist that tables are put in a table-environment?}

\begin{table}
 \centering
 \begin{tabular}[t]{lr}
  \toprule
  Type of language                     & Number \\
  \midrule
  No articles at all                   & 198    \\
  Both indefinite and definite article & 154    \\
  Definite, but no indefinite article  & 89     \\
  Indefinite, but no definite article  & 40     \\
  \midrule
  Total                                & 481    \\
  \bottomrule                                       
\end{tabular}
  \caption{Article Systems in the World's Languages according to \emph{WALS}}
  \label{tab:article-systems}
\end{table}


%%%%%%%%%%
%% The official end of this file
%%%%%%%%%%
