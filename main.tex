
% AUTHOR_master.tex
% Prepared by C. Pinon (cjpinon@implicature.xyz) for EISS 13 (see
% https://implicature.xyz/eiss13/)
% 2020-12-30
% 2020-02-02

% This is the master file for your paper. This is the file that you
% should compile.

\documentclass[a5paper,12pt,twoside]{article}% Please don't change

\usepackage{eiss}% Please don't change

% For your information, eiss.sty explicitly requires the following
% publicly available packages (most of which are probably already
% installed on recent LaTeX systems). Please don't reload these
% packages!

% (Note that you can latex the file check-for-packages.tex in order to
% check whether these packages are installed on your system.)

% cabin
% calc
% caption
% datetime
% draftwatermark
% enumitem
% fancyhdr
% fontenc
% fourier
% geometry
% graphicx
% ifpdf
% natbib
% quoting
% textcomp
% theorem
% titlesec
% url
% xcolor
% zlmtt

%%%%%%%%%%
%% In order to use this file as intended, you need to complete the
%% following 12 steps. (It should go quickly!)
%%%%%%%%%%

%%%%%%%%%%
%% Step 1 (The font)
%%%%%%%%%%

% The official font for EISS 13 is Utopia, which is preselected:

\UseUtopia

% Alternative fonts are Latin Modern or Times:

% \UseLatinModern
% \UseTimes

% (Don't select Latin Modern or Times for EISS 13 unless the use of
% Utopia presents a problem. Note that ONLY ONE of these three font
% options should be selected.)

%%%%%%%%%%
%% Step 2 (The input encoding)
%%%%%%%%%%

% Note: If you're using a version of LaTeX from 2018/04/01 or later,
% then UTF-8 is the default input encoding and it's no longer necessary
% to set the input encoding to utf8 explicitly.

% If you're directly inserting only ASCII (basically: non-accented)
% characters, then you don't have to explicitly set an input encoding,
% and you can go to step 3! If you're directly inserting some non-ASCII
% (accented) characters, then you have to choose an input
% encoding. Nowadays, the standard choice of non-ASCII input encoding is
% utf8. If you want to use this or another input encoding, uncomment the
% following line and specify the encoding that you want to use (and then
% make sure that you save your file in this encoding):

% \usepackage[utf8]{inputenc}

% (Some popular legacy encodings are latin1, latin2, latin9, cp1252, and
% cp1250.)

%%%%%%%%%%
%% Step 3 (The title; there are potentially three substeps)
%%%%%%%%%%

%%%%%%%%%%
%% Step 3a (The title of the paper on the first page)
%%%%%%%%%%

% Uncomment the following line and specify the title of your paper
% (forced line breaks with "\\" are allowed but not encouraged):

\renewcommand{\TitleOfPaper}{This Title Needs Changed}

%%%%%%%%%%
%% Step 3b (The title of the paper in the PDF metadata)
%%%%%%%%%%

% If the title of the paper on the first page (specified in step 3a)
% contains "\\" or an emphasized/italicized expression, then uncomment
% the following line and specify the title of the paper without "\\" and
% without an emphasized/italicized expression:

% \renewcommand{\TitleOfPaperpdfinfo}{The title of the paper without "\\" or emphasis/italics}

% (If the title of the paper specified in step 3a doesn't contain "\\"
% or emphasis/italics, then there's no need to set this and you can go
% to step 3c!)

%%%%%%%%%%
%% Step 3c (The title of the paper in the running headers, if different)
%%%%%%%%%%

% If the title of the paper in the running headers differs from the
% title of the paper on the first page (specified in step 3a), then
% uncomment the following line and specify the title of the paper in the
% headers (note that no forced line breaks are allowed here):

% \renewcommand{\TitleOfPaperInHeader}{The title in the headers}

% (Normally, the title of the paper in the running headers is identical
% to the title of the paper on the first page, in which case there's no
% need to set this. But sometimes, the title of the paper in the running
% headers needs to be shortened a bit, in which case the shortened title
% should be specified here.)

%%%%%%%%%%
%% Step 4 (The authors; there are up to four substeps, depending on the
%% number of authors)
%%%%%%%%%%

%%%%%%%%%%
%% Step 4a (The FIRST author)
%%%%%%%%%%

% Uncomment the following line and specify the first name of the first
% author:

\renewcommand{\AuthorAFirstName}{Anonymous}

% Uncomment the following line and specify the first initial (letter) of
% the first name of the first author:

\renewcommand{\AuthorAFirstInitial}{A}

% In certain cases, it may be more appropriate to give the first initial
% as a combination of two letters: an upper case letter followed
% immediately by a lower case letter. For example, if the first name is
% "Szilvia", the first initial would best be "Sz" (without the quotes,
% of course).

% Uncomment the following line and specify the last (family) name of the
% first author:

\renewcommand{\AuthorALastName}{Coward}

% Uncomment the following line and specify the affiliation of the first
% author (just the name of the institution):

\renewcommand{\AuthorAAffiliation}{Utopian University of Atlantis}

% Uncomment one of the following lines and specify EITHER the web page
% OR the email address of the first author (but not both; a web page is
% preferred).

% NOTE: If you want to use a tilde (~), put "\noexpand~" instead of "~"
% (without the quotes). This applies to the URLs of the other authors as
% well.

% \renewcommand{\AuthorAUrl}{http://web-page-author-1/}
\renewcommand{\AuthorAUrl}{anonymous.coward@utopian-atlantis.edu}

%%%%%%%%%%
%% Step 4b (The SECOND author, if there is one; if not, go to step 5!)
%%%%%%%%%%

% Uncomment the following line and specify the first name of the second
% author:

% \renewcommand{\AuthorBFirstName}{First-name-author-2}

% Uncomment the following line and specify the first initial (letter) of
% the first name of the second author:

% \renewcommand{\AuthorBFirstInitial}{First-initial-first-name-author-2}

% Uncomment the following line and specify the last (family) name of the
% second author:

% \renewcommand{\AuthorBLastName}{Last-name-author-2}

% Uncomment the following line and specify the affiliation of the second
% author (just the name of the institution):

% \renewcommand{\AuthorBAffiliation}{Affiliation-author-2}

% Uncomment one of the following lines and specify EITHER the web page
% OR the email address of the second author (but not both; a web page is
% preferred).

% \renewcommand{\AuthorBUrl}{http://web-page-author-2/}
% \renewcommand{\AuthorBUrl}{author-2@university.edu}

%%%%%%%%%%
%% Step 4c (The THIRD author, if there is one; if not, go to step 5!)
%%%%%%%%%%

% Uncomment the following line and specify the first name of the third
% author:

% \renewcommand{\AuthorCFirstName}{First-name-author-3}

% Uncomment the following line and specify the first initial (letter) of
% the first name of the third author:

% \renewcommand{\AuthorCFirstInitial}{First-initial-first-name-author-3}

% Uncomment the following line and specify the last (family) name of the
% third author:

% \renewcommand{\AuthorCLastName}{Last-name-author-3}

% Uncomment the following line and specify the affiliation of the third
% author (just the name of the institution):

% \renewcommand{\AuthorCAffiliation}{Affiliation-author-3}

% Uncomment one of the following lines and specify EITHER the web page
% OR the email address of the third author (but not both; a web page is
% preferred).

% \renewcommand{\AuthorCUrl}{http://web-page-author-3/}
% \renewcommand{\AuthorCUrl}{author-3@university.edu}

%%%%%%%%%%
%% Step 4d (The FOURTH author, if there is one; if not, go to step 5!)
%%%%%%%%%%

% Uncomment the following line and specify the first name of the fourth
% author:

% \renewcommand{\AuthorDFirstName}{First-name-author-4}

% Uncomment the following line and specify the first initial (letter) of
% the first name of the fourth author:

% \renewcommand{\AuthorDFirstInitial}{First-initial-first-name-author-4}

% Uncomment the following line and specify the last (family) name of the
% fourth author:

% \renewcommand{\AuthorDLastName}{Last-name-author-4}

% Uncomment the following line and specify the affiliation of the fourth
% author (just the name of the institution):

% \renewcommand{\AuthorDAffiliation}{Affiliation-author-4}

% Uncomment one of the following lines and specify EITHER the web page
% OR the email address of the third author (but not both; a web page is
% preferred).

% \renewcommand{\AuthorDUrl}{http://web-page-author-4/}
% \renewcommand{\AuthorDUrl}{author-4@university.edu}

%%%%%%%%%%
%% Step 5 (Linguistic examples)
%%%%%%%%%%

% If you want to use phonetic (IPA) symbols in your linguistic examples,
% uncomment the following line to use the package tipa:

% \UseTipa

% linguex and expex are two well-known packages for formatting
% linguistic examples. If you wish to use one of these, uncomment the
% appropriate line below:

% \UseLinguex

% \UseExpex

% If your paper contains many glossed examples, you may choose expex
% with a smaller font size instead in order to save some space:

% \UseExpexSmall

% But note that standard choice is \UseExpex and not \UseExpexSmall.

% If you prefer to use a different package for formatting linguistic
% examples, please specify it in the next step.

%%%%%%%%%%
%% Step 6 (Packages or definitions that you need)
%%%%%%%%%%

% Include here any \usepackage statements or definitions that you need
% for this paper, but please load only those packages that you REALLY
% NEED (and recall the list of already loaded packages given
% above). Note that the fonts, margins, etc. have already been set, so
% please don't try to override these.

% Please don't use hyperref! But remember to use \url{} for any URLs in
% your main text!

% Some suggestions:

% For extra LaTeX math symbols:

% \usepackage{latexsym}

% A well-known package for drawing syntactic trees:

% \usepackage[nocenter]{qtree}

% Add packages and definitions (again, only those that you really need
% for this paper!):

% \usepackage{}

% \newcommand{}{}

%%%%%%%%%%
%% Nothing more to do here -- go to step 7!
%%%%%%%%%%

% Eventually we'll add a visible logo (please don't change):
% \renewcommand{\EISSlogo}{\includegraphics{eiss13_logo}}

% For the watermarks (until the final version):
\usepackage[firstpage]{draftwatermark}% Please don't change
% \usepackage{draftwatermark}%
\SetWatermarkFontSize{2.6cm}%
\SetWatermarkText{\shortstack{Draft\\ \today}}%
% \SetWatermarkText{\shortstack{Example\\ \today}}%
% \SetWatermarkText{\shortstack{Preview\\ \today}}

%% Set up the bibliography:
\bibliography{AUTHOR_bib.bib}

\begin{document}% Please don't change

\label{PageFirst}% Please don't change

% We'd prefer not to put potentially misleading page numbers in the
% reference information on the first page until we know what the final
% page numbers will be, so for the time being we'll use 0 to 00:

\renewcommand{\FirstPage}{0}% Please don't change
\renewcommand{\LastPage}{00}

% Once we know which page your paper begins on, we'll uncomment the
% following line and set the value accordingly (please don't change):

% \setcounter{page}{137}

%%%%%%%%%%
%% Step 7 (The abstract)
%%%%%%%%%%

% Uncomment the following line and give your abstract (if necessary,
% your abstract may contain more than one paragraph, but please try to
% make it neither too long nor too short!):

\renewcommand{\AbstractOfPaper}{The abstract of your paper goes here.}

%%%%%%%%%%
%% Step 8 (The keywords; there are potentially three substeps)
%%%%%%%%%%

%%%%%%%%%%
%% Step 8a (Specify the keywords)
%%%%%%%%%%

% Uncomment at least the first FOUR of the following lines and specify
% your keywords. Please use AT LEAST FOUR and AT MOST EIGHT
% keywords. (Yes, this is kind of strict, sorry!)

\renewcommand{\KeywordOne}{Change}
\renewcommand{\KeywordTwo}{These}
\renewcommand{\KeywordThree}{Keywords}
\renewcommand{\KeywordFour}{Pulease}
% \renewcommand{\KeywordFive}{Keyword5}
% \renewcommand{\KeywordSix}{Keyword6}
% \renewcommand{\KeywordSeven}{Keyword7}
% \renewcommand{\KeywordEight}{Keyword8}

%%%%%%%%%%
%% Step 8b (Specify the keywords in the PDF metainfo)
%%%%%%%%%%

% IF any of the keywords just specified (in Step 8a) is
% EMPHASIZED/ITALICIZED, then give those keyword(s) again by
% uncommenting the relevant line(s) below and specifying those
% keyword(s) without emphasis/italics. If none of your keywords are
% emphasized/italicized, there's no need to set this and you can go to
% step 8c!

% \renewcommand{\KeywordOnepdfinfo}{Keyword1}
% \renewcommand{\KeywordTwopdfinfo}{Keyword2}
% \renewcommand{\KeywordThreepdfinfo}{Keyword3}
% \renewcommand{\KeywordFourpdfinfo}{Keyword4}
% \renewcommand{\KeywordFivepdfinfo}{Keyword5}
% \renewcommand{\KeywordSixpdfinfo}{Keyword6}
% \renewcommand{\KeywordSevenpdfinfo}{Keyword7}
% \renewcommand{\KeywordEightpdfinfo}{Keyword8}

%%%%%%%%%%
%% Step 8c (Specify the number of keywords)
%%%%%%%%%%

% Uncomment exactly one of the following lines, depending on how many
% keywords you specified:

\UseFourKeyword
% \UseFiveKeyword
% \UseSixKeyword
% \UseSevenKeyword
% \UseEightKeyword

%%%%%%%%%%
%% Step 9 (The acknowledgments)
%%%%%%%%%%

% Uncomment the following line and give your acknowledgments:

% \renewcommand{\Thanks}{Your acknowledgments go here.}

%%%%%%%%%%
%% Step 10 (Make the title, authors, etc.)
%%%%%%%%%%

% Uncomment exactly one of the following lines, depending on the number
% of authors you specified:

\MakeOneAuthor
% \MakeTwoAuthor
% \MakeThreeAuthor
% \MakeFourAuthor

%%%%%%%%%%
%% Step 11 (Input the author body-file)
%%%%%%%%%%

% Uncomment the following line and specify the name of the author
% body-file to be inserted:


% AUTHOR_body.tex
% Prepared by C. Pinon (cjpinon@implicature.xyz) for EISS 13 (see
% https://implicature.xyz/eiss13/)
% 2020-02-02

% Please use this file (which you should rename) for the main ("body")
% part of your paper. Be sure to compile the master file, not this one.

% Put your main text here. Recall that the references are in a separate
% file. And recall also that "\begin{document}" and "\end{document}"
% have already been inserted, which means that you can begin immediately
% with the first section.

\section{Introduction}\label{sec:1}

In this paper, we will show how semantics and pragmatics reduce to
syntax (or the other way around, see \citealt{Kamp:1973,Searle:1964}) \ldots 

Yes you can śačĉḥ -- ţŷpe şōmê wild Üñïçôđė.

For examples, use \textbf{linguex}, and for glossing morphological categories, please refer to the Leipzig Glossing Rules.

\exg. 
win er der Namen Gottes het usgsprochn-a ghabe\\
when he the name.\textsc{m.sg} God.\textsc{gen} has pronounced-\textsc{m.sg} had\\
%\glt 
`once he had pronounced the name of God'

For examples with several subexamples:

\ex. \label{ex:1}\a. \label{ex:2}First example
\b. Second
\bg. Dies ist ein schönes Beispiel mit einer Glose.\\
This is a beautiful example with a gloss.\\
\a. `Some translation'
\b. `Some other'
\z.
\b. Fourth

You can refer back to examples like \ref{ex:1} and \ref{ex:2}.

In order to make trees, use the \textbf{forest} package.

% LaTeX doesn't care how you arrange the tree in the source code, but for your won sake, you want probably want to write something as follows:
\ex. \begin{forest}
[S 
  [DP [D [ Ce\\\footnotesize{This} ] ]
      [NP [AP [A [ petit\\\footnotesize{small} ] ] ]
          [N [ arbre\\\footnotesize{tree} ] ] ] ]
  [VP [V [ est\\\footnotesize{is} ] ]
      [AP [AdvP [Adv [ très\\\footnotesize{very} ] ] ]
          [A [ joli\\\footnotesize{nice} ] ] ] ] ]
\end{forest}

You can also make a more involved example, with arrows:

\ex.  \label{ex:rp-structure} 
\footnotesize{%
\begin{forest} for tree={l=10mm, l sep=0}
   [AuxP, name=top 
      [beP [RP 
              [DP [John] ] 
              [R$'$ [withP  
                       [RP [DP [hair] ] 
                           [R$'$ [AP [VP ] 
                                     [A [ colored, name=cellar ] ] ] 
                                 [R] ] ] 
                        [with, name=source] ] 
                    [R, name=c1] ] ] 
            [be, name=c2]   ]  
      [Aux [ has ] ] ]
    % and now, we draw some arrows  
   \draw[->,color=red,very thick,dotted] (source) to[out=east,in=south east] (c1);
   \draw[->,color=blue,ultra thick,dashed] (c1) to[out=east,in=south east] (c2);
   \draw[-{Latex[length=2.5mm]},color=purple,ultra thick] (cellar) to[out=east,in=east]  
                        node [below, sloped] (TextNode2) {because why the hell not} (top);
\end{forest}}

\citet[451]{Szadrowsky:1936} claims something, but I am not sure whether \citet[70]{Kamp:1973} would agree (or \citealt[50]{Searle:1964}, by that matter).

For avms, we use \textbf{langsci-avm}.

\ex. \avm{[ attr1 & \1\\
attr2 & \2[attr3 & val3\\
attr4 & val4] ]}

You can combine \textbf{forest} and \textbf{langsci-avm}:

\ex. \begin{forest}
[A [B] [{\avm{[attr1 & val1\\
attr2 & val2\\
attr3 & val3]}} ] ]
\end{forest}l

And finally, for tables, we use the \textbf{booktabs}-environment, as illustrated in table \ref{tab:article-systems}. 



\textbf{Question: do we insist that tables are put in a table-environment?}

\begin{table}
 \centering
 \begin{tabular}[t]{lr}
  \toprule
  Type of language                     & Number \\
  \midrule
  No articles at all                   & 198    \\
  Both indefinite and definite article & 154    \\
  Definite, but no indefinite article  & 89     \\
  Indefinite, but no definite article  & 40     \\
  \midrule
  Total                                & 481    \\
  \bottomrule                                       
\end{tabular}
  \caption{Article Systems in the World's Languages according to \emph{WALS}}
  \label{tab:article-systems}
\end{table}


%%%%%%%%%%
%% The official end of this file
%%%%%%%%%%


%%%%%%%%%%
%% Nothing more to do here -- go to step 12!
%%%%%%%%%%

% The acknowledgments appear just before the references:

\MakeThanks% Please don't change

%%%%%%%%%%
%% Step 12 (Input the author refs-file)
%%%%%%%%%%

% Uncomment the following line and specify the name of the author
% refs-file to be inserted:


% AUTHOR_refs.tex
% Prepared by C. Pinon (cjpinon@implicature.xyz) for EISS 13 (see
% https://implicature.xyz/eiss13/)
% 2020-02-02

% Please use this file (which you should rename) for the references of
% your paper. Be sure to compile the master file, not this one.

% Important: Since we're using natbib, you should cite references in
% your main text in the way required by natbib (e.g. \citet{},
% \citealt{}, etc.). Consult the natbib documentation (natnotes.pdf,
% natbib.pdf) for this.

% The preferred way to format your references is to use BibTeX. For
% this, please use the BibTeX style file glossa.bst (from Glossa).

%\bibliographystyle{glossa}

% Uncomment the following line and specify the name of your BIB file
% (without the suffix .bib):

%\bibliography{AUTHOR_bib}

\printbibliography

% Please be sure to always use an n-dash (--) for page intervals in your
% BIB file. For example, write "1--24" and not "1-24".

%%%%%%%%%%
%% The official end of this file
%%%%%%%%%%


%%%%%%%%%%
%% Nothing more to do in this file! :-)
%%%%%%%%%%

\label{PageLast}% Please don't change

\end{document}% Please don't change

%%%%%%%%%%
%% The official end of this file
%%%%%%%%%%
